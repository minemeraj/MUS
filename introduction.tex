\section{Introduction}
\label{sec:introduction}
Many real-world problems arising for example in formal verification of hardware and software can be formulated as propositional logic formulas, which are Boolean combinations of atomic propositions. When transformed to conjunctive normal form (CNF), i.e., when formulas are conjunctions of disjunctions of propositions (clauses) and negated propositions (literals). Boolean satisfiability (SAT) solvers can be employed to determine whether a formula is satisfiable or unsatisfiable i.e., whether there exists an assignment of Boolean values for the propositions such that the formula evaluates to true. When a formula is unsatisfiable, it is required to find minimal unsatisfiable subsets of the problem's clauses. Minimal Unsatisfiable subsets (MUSes), also called Minimal Unsatisfiable Cores (MUCs), are essential to determine the reasons for the failure of the system. In the past few years, the interest and research on the extraction of MUSes of an unsatisfiable constraint system has been increasing, but most of the research has concentrated on the algorithms to extract a single MUS \cite{karem}. This single MUS may not provide complete information about the failure of constraint system, because the system might have multiple MUSes and the appearance of any of them makes the system unsatisfiable.\newline
In this paper, we explaine some algorithms which are necessary to compute all MUSs of a system as in \cite{karem}. The paper is organized as follows. Section $2$ presents some preliminaries with formal definitions. We introduce the algorithms in details in Section $3$ and conclude in Section $4$.