\section{Introduction}
\label{sec:introduction}
Quantifier/ variable elimination for elementary real algebra is a fundamental problem. This problem can be solved easily by using virtual solution. In 1993, the concept of virtual substitution was first introduced. Initially it was a procedure to eliminate quantifier/variable elimination for linear real arithmetic formulas.Further, virtual substitution became a procedure of quantifier elimination for non-linear arithmetic formulas.Virtual substitution cannot eliminate quantified variables whose degree is higher than 2.

Section 2 consists some preliminaries with some definitions. How to construct the real zeros is explained in section 3. In the next section we will know some substitution rules by which we can eliminate variables from a formula with an example. In the section 4 an idea to eliminate quantifiers is explained by an example and conclude in the last section.
