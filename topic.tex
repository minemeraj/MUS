\section{Algorithms for Computing Minimal Unsatisfiable Subset}
\label{sec:algorithms for computing minimal unsatisfiable subset}
The approach of computing all MUSes of $\varphi$ is to first find all MCSes($\varphi$) and then to compute all irreducible hitting sets of MCSes($\varphi$), which are all MUSes($\varphi$). So, there are two phases to generate all MUSes of a unsatisfiable CNF formula. The first phase is to compute MCSes by using an algorithm and the second phase is to compute MUSes from the MCSes by using a recursive algorithm which authors have developed to compute irreducible hitting sets.\newline
In the first phase, authors use a SAT solver (produces an satisfying assignment, if exists) to work with the input given and provide hitting sets of MUSes without revealing the underlying MUSes. For the second phase, they get all the information with MCSes and they use an recursive algorithm to generate minimal and irreducible hitting sets. 

$$algorithmMCSLikhteHobe$$
\subsection{Computing MCSes}
Algorithm $1$ finds MCSes by solving a CNF formula $\varphi$. The goal is to find minimal set of clauses which makes $\varphi$ satisfiable. In Line $1$, $\varphi$ is augmented with clause-selector variables and create $\varphi^{\prime}$. By augmenting $\varphi$ increase the number of variables in $\varphi$. Also the search space grows for finding a satisfying assignment. Minimal set of variables $w_{i}$ are assigned to $false$ so that minimum number of clauses are disabled to finding the satisfying assignment of $\varphi^{\prime}$. The set of variables $w_{i}$ are assigned to $false$ indicates the clauses as an MCS.\newline
Each iteration of the outer while loop (lines $4$-$12$) finds an MCS of size $k$, which is incremented by $1$ after each iteration. In Line $5$, a clause is added of the form $atMostFormula\clubsuit\clubsuit$ to $\varphi^{\prime}$ and a new formula $\varphi^{\prime}_{\text k}$ is created.
\subsection{Compuitng MUSes}
