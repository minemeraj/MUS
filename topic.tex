\section{Topic}
\label{sec:topic}
This section concerns the main topic. In the following you can see a
small illustration of how to use itemizings and enumerations.

\begin{itemize}
	\item Point 1.
	\item Point 2.
\end{itemize}

\begin{enumerate}
	\item Point 1.
	\item Point 2.
	\begin{enumerate}[I)]
  \item Point 1.
  \item Point 2.
\end{enumerate}
\end{enumerate}

\begin{enumerate}
	\item Point 1.
	\item Point 2.
\end{enumerate}

\begin{description}
	\item[Term one: ] Description of term one.
	\item[Term two: ] Description of term two.
\end{description}

In Algorithm~\ref{alg:nameOfAlgorithm} you can see how we define an
algorithm.

\begin{Algorithm}
	\caption{Describe the purpose of the algorithm. For more 
	information see the \href{../manuals/newalg.pdf}{newalg-
	Manual}.}
	\label{alg:nameOfAlgorithm}
	\begin{algorithm}{\text{void method}}{\text{typeA argumentA, 
												typeB argumentB}}
		\text{write the algorithm in pseudocode} \\
		\text{it should not go into detail, but display main idea} \\
		\text{however, keep being consistent} \\
		x\=1\text{ (this is how to assign a value to a variable)} \\
		\begin{WHILE}{\text{a condition being True or False}}
			\text{do something} \\
			\text{and something else} \\
		\end{WHILE} \\
		\begin{IF}{\text{a condition being True or False}}
			\text{point }1 \\
		\ELSE
			\begin{IF}{\text{another condition}}
				\text{point }2 \\
			\end{IF}
		\ELSE
			\text{point }3 \\
		\end{IF}
		\RETURN True
	\end{algorithm}
\end{Algorithm}

\subsection{Example}
Give an example to illustrate the idea of your topic. Import images
in the following way. Store the images in a separate folder as 
precasted in our template.

\begin{figure}[htb]
	\begin{center}
		\includegraphics[width=0.8\linewidth]{pictures/ratte.jpg}
	\end{center}
	\caption{Proseminar supervisor's pet.}
	\label{fig:rat}
\end{figure}
